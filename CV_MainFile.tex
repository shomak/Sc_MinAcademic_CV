%%%%%%%%%%%%%%%%%%%%%%%%%%%%%%%%%%%%%%%%%%%%%%%%%%%%%%%%%%%%%%%%%%%%%%%%%%%%%%%%%%%%%%%%%%%%%%%%%%%%%%%
% MINIMALIST ACADEMIC CV CLASS 
% CREATED BY SHOMAK CHAKRABARTI, PENN STATE UNIVERSITY
% LAST EDITED ON JULY 20, 2020
%Accessible at https://github.com/shomak/Sc_MinAcademic_CV.git

%The default path has been written for a MacOS system. For Windows user, just change the / to \ i.e. change the path to the class file to: sty\CV_Class
\documentclass{sty/CV_Class}
\begin{document}

%%%%%%%%%%%%%%%%%%%%%%%%%%%%%%%%%%%%%%%%%%%%%%%%%%%%%%%%%%%%%%%%%%%%%%%%%%

%-> Add the month-year format that appears at the bottom left corner of every page.  
\footerdate{Jul 2020}

%%%%%%%%%%%%%%%%%%%%%%%%%%%%%%%%%%%%%%%%%%%%%%%%%%%%%%%%%%%%%%%%%%%%%%%%%%

%-> Format: enter first name and last name within the first box.
%-> Format: enter email id in first box; personal webpage in 2nd box - do not include the term http:// when writing your webpage. It is automatically added.
\name{Firstname Lastname}  
\contact{email@domain.com}{webpage.com} 

%%%%%%%%%%%%%%%%%%%%%%%%%%%%%%%%%%%%%%%%%%%%%%%%%%%%%%%%%%%%%%%%%%%%%%%%%%

%-> Format: command: \employment{}{}{}{}{} 
%-> Add multiple employments by adding \item to the \itemize command. Only use the \\~\\ option at the at the end of the LAST \item to increase space between the options. 
%-> If you want to leave one box empty, make sure you still type {} - just do not write anything within the brackets. 
\section{\texttt{EMPLOYMENT}}
\begin{itemize}
	\item \employment{Position}{University}{Country}{Start}{End}\\~\\
\end{itemize}

%%%%%%%%%%%%%%%%%%%%%%%%%%%%%%%%%%%%%%%%%%%%%%%%%%%%%%%%%%%%%%%%%%%%%%%%%%

%-> Format: command: \education{}{}{}{}{} 
%-> Add multiple degrees by adding \item to the \itemize command. Only use the \\~\\ option at the at the end of the LAST \item to increase space between the options. 
%-> If you want to leave one box empty, make sure you still type {} - just do not write anything within the brackets. 
\section{\texttt{EDUCATION}}
\begin{itemize}
	\item \education{Degree}{university}{Country}{Start}{End}\\~\\
\end{itemize}

%%%%%%%%%%%%%%%%%%%%%%%%%%%%%%%%%%%%%%%%%%%%%%%%%%%%%%%%%%%%%%%%%%%%%%%%%%

%-> Format: table with column 1 as topic name, column 2 as response.
%-> Topics - Main areas of expertise (ROW 1), Computer Skills (ROW 2), Language Skills (ROW 3)
%-> modify the gaps between the items by changing [0.5em]. Only use the \\~\\ option at the at the end of the LAST \item to increase space between the options. 
%-> Uncomment the center commands if you want to amke the table in the middle
\section{\texttt{INTERESTS \& SKILLS}}
%\begin{center}
\begin{tabular}{ l l }
 \textit{Fields of Interest}: & Subject \\ [0.5em]
 \textit{Computer Skills}: & \LaTeX \\ [0.5em]
 \textit{Language Skills}: & English (\textit{proficient}).\\~\\     
\end{tabular}
%\end{center}

%%%%%%%%%%%%%%%%%%%%%%%%%%%%%%%%%%%%%%%%%%%%%%%%%%%%%%%%%%%%%%%%%%%%%%%%%%

%-> Format: \paper{}{}{}{}
%-> Foramt: \publication{}{} - first box is journal name, second box is link to the paper (dont add http://)
%-> You can include the brackets when mentioning co-authors. 
%-> use \item with \itemize for subsection heading; \item with \enumerate to list the papers
%-> Two options: you can put everything inside an itemized list - in that case comment the lines \textit{} and uncomment the \begin{itemize}, \item\textit{} and \end{itemize} lines
\section{\texttt{RESEARCH PAPERS}}
%\begin{itemize}
	%\item \textit{Published Paper}
\textit{Published Papers}:
	\begin{enumerate}
		\item \paper{Papername}{Date}{(with co-author)}
		\begin{itemize}
			\item[] \publication{journalname}{journalname.com}\\
		\end{itemize} 
	\end{enumerate}
	%\item \textit{Working papers}
\textit{Working Papers}:
	\begin{enumerate}
		\item \paper{Papername}{Date}{(with co-author)}
		\begin{itemize}
			\item[] \publication{wpseries archive}{ssrn.com}\\~\\
		\end{itemize}  
	\end{enumerate}
%\end{itemize} 

%%%%%%%%%%%%%%%%%%%%%%%%%%%%%%%%%%%%%%%%%%%%%%%%%%%%%%%%%%%%%%%%%%%%%%%%%%

%-> Format: \rawork{}{}{}{} and |tawork{}{}{}{}
%-> For course instructor list, you leave the first box {} blank 
%-> you can remove the [] from\item[] if you want numbered/bulleted lists.
\section{\texttt{TEACHING \& RESEARCH ACTIVITIES}}
\begin{itemize}
	\item \textit{Course Instructor}
	\begin{itemize}
		\item[] \tawork{}{Course name}{Start}{End}
	\end{itemize}
	\item \textit{Teaching Assistant for}
	\begin{itemize}
		\item[] \tawork{Instructor: }{Course name}{Start}{End}
	\end{itemize}
	\item \textit{Reseach Assistant for}
	\begin{itemize}
		\item[] \rawork{Professor}{ University}{Start}{End}\\~\\
	\end{itemize}
\end{itemize}

%%%%%%%%%%%%%%%%%%%%%%%%%%%%%%%%%%%%%%%%%%%%%%%%%%%%%%%%%%%%%%%%%%%%%%%%%%

%-> Format: table with column 1 as Activity Name, column 2 as response.
%-> Activity Name - activities performed
%-> modify the gaps between the items by changing [0.5em]. Only use the \\~\\ option at the at the end of the LAST \item to increase space between the options. 
%-> Uncomment the center commands if you want to amke the table in the middle
\section{\texttt{PROFESSIONAL ACTIVITIES}}
%\begin{center}
\begin{tabular}{ l l }
 \textit{Referee}: & Journals \\ [0.5em]
 \textit{Position}: & University \\~\\
\end{tabular}
%\end{center}

%%%%%%%%%%%%%%%%%%%%%%%%%%%%%%%%%%%%%%%%%%%%%%%%%%%%%%%%%%%%%%%%%%%%%%%%%%

%-> Format: table with column 1 as Universities where presentation was done, column 2 as Month-Year of presentation.
%-> Activity Name - activities performed
%-> Write only 3 letters for months e.g. Jan, Feb, Mar etc
%-> modify the gaps between the items by changing [0.5em]. Only use the \\~\\ option at the at the end of the LAST \item to increase space between the options. 
%-> Uncomment the center commands if you want to amke the table in the middle
\section{\texttt{PRESENTATIONS}}
%\begin{center}
\begin{tabular}{ l l }
 \textit{Universities}: &  \hfill month-year \\ [0.5em]
 \textit{Universities}: & \hfill month-year \\~\\
\end{tabular}
%\end{center}

%%%%%%%%%%%%%%%%%%%%%%%%%%%%%%%%%%%%%%%%%%%%%%%%%%%%%%%%%%%%%%%%%%%%%%%%%%

%-> Format: \Honours{}{}{}
%-> Only use the \\~\\ option at the at the end of the LAST \item to increase space between the options. 
\section{\texttt{HONOURS}}
\begin{itemize}
	\item \honours{Award name}{University}{Year}\\~\\
\end{itemize}

%%%%%%%%%%%%%%%%%%%%%%%%%%%%%%%%%%%%%%%%%%%%%%%%%%%%%%%%%%%%%%%%%%%%%%%%%%

%-> Format: tabular with each column for a different professor 
%-> ROW 1: professor name; ROW 2: Department; ROW 3: University; ROW 4: email-id
%-> Uncomment the center commands if you want to amke the table in the middle
\section{\texttt{REFERENCES}}
%\begin{center}
\begin{tabular}{l l l}
\textbf{\texttt{Professor 1}} & \textbf{\texttt{Professor 2}} & \textbf{\texttt{Professor 3}} \\
Department  & Department & Department\\
University & University& University\\
\href{mailto:prof1@domain.edu}{\texttt{professor1@domain.edu}} & \href{mailto:prof2@domain.edu}{\texttt{professor2@domain.edu}} & \href{mailto:prof3@domain.edu}{\texttt{professor3@domain.edu}}\\~\\
\end{tabular}
%\end{center}

%%%%%%%%%%%%%%%%%%%%%%%%%%%%%%%%%%%%%%%%%%%%%%%%%%%%%%%%%%%%%%%%%%%%%%%%%%

\end{document}

%%%%%%%%%%%%%%%%%%%%%%%%%%%%%%%%%%%%%%%%%%%%%%%%%%%%%%%%%%%%%%%%%%%%%%%%%%
